\documentclass[]{article}
\usepackage{lmodern}
\usepackage{amssymb,amsmath}
\usepackage{ifxetex,ifluatex}
\usepackage{fixltx2e} % provides \textsubscript
\ifnum 0\ifxetex 1\fi\ifluatex 1\fi=0 % if pdftex
  \usepackage[T1]{fontenc}
  \usepackage[utf8]{inputenc}
\else % if luatex or xelatex
  \ifxetex
    \usepackage{mathspec}
  \else
    \usepackage{fontspec}
  \fi
  \defaultfontfeatures{Ligatures=TeX,Scale=MatchLowercase}
\fi
% use upquote if available, for straight quotes in verbatim environments
\IfFileExists{upquote.sty}{\usepackage{upquote}}{}
% use microtype if available
\IfFileExists{microtype.sty}{%
\usepackage{microtype}
\UseMicrotypeSet[protrusion]{basicmath} % disable protrusion for tt fonts
}{}
\usepackage[margin=1in]{geometry}
\usepackage{hyperref}
\hypersetup{unicode=true,
            pdftitle={Removing Embedded Nulls with Bash},
            pdfauthor={Amy Linehan},
            pdfborder={0 0 0},
            breaklinks=true}
\urlstyle{same}  % don't use monospace font for urls
\usepackage{graphicx,grffile}
\makeatletter
\def\maxwidth{\ifdim\Gin@nat@width>\linewidth\linewidth\else\Gin@nat@width\fi}
\def\maxheight{\ifdim\Gin@nat@height>\textheight\textheight\else\Gin@nat@height\fi}
\makeatother
% Scale images if necessary, so that they will not overflow the page
% margins by default, and it is still possible to overwrite the defaults
% using explicit options in \includegraphics[width, height, ...]{}
\setkeys{Gin}{width=\maxwidth,height=\maxheight,keepaspectratio}
\IfFileExists{parskip.sty}{%
\usepackage{parskip}
}{% else
\setlength{\parindent}{0pt}
\setlength{\parskip}{6pt plus 2pt minus 1pt}
}
\setlength{\emergencystretch}{3em}  % prevent overfull lines
\providecommand{\tightlist}{%
  \setlength{\itemsep}{0pt}\setlength{\parskip}{0pt}}
\setcounter{secnumdepth}{0}
% Redefines (sub)paragraphs to behave more like sections
\ifx\paragraph\undefined\else
\let\oldparagraph\paragraph
\renewcommand{\paragraph}[1]{\oldparagraph{#1}\mbox{}}
\fi
\ifx\subparagraph\undefined\else
\let\oldsubparagraph\subparagraph
\renewcommand{\subparagraph}[1]{\oldsubparagraph{#1}\mbox{}}
\fi

%%% Use protect on footnotes to avoid problems with footnotes in titles
\let\rmarkdownfootnote\footnote%
\def\footnote{\protect\rmarkdownfootnote}

%%% Change title format to be more compact
\usepackage{titling}

% Create subtitle command for use in maketitle
\providecommand{\subtitle}[1]{
  \posttitle{
    \begin{center}\large#1\end{center}
    }
}

\setlength{\droptitle}{-2em}

  \title{Removing Embedded Nulls with Bash}
    \pretitle{\vspace{\droptitle}\centering\huge}
  \posttitle{\par}
    \author{Amy Linehan}
    \preauthor{\centering\large\emph}
  \postauthor{\par}
      \predate{\centering\large\emph}
  \postdate{\par}
    \date{2020-07-31}


\begin{document}
\maketitle

In most cases, we can rely on R, Python or another programming language
to clean and organize our data. In some instances, we may also need to
utilize an OCR software or manual updates to prepare data for analysis.
However, sometimes I've come across data that I can't read into R or
Python accurately and can't find a free or available OCR software to
extract the data into a more manageable format. In these cases, I have
found Bash to be a reliable friend.

If you've never used bash before, for Windows users, I recommend
downloading git bash: \url{https://gitforwindows.org/} . I also use
Visual Studios code to write and review my scripts:
\url{https://code.visualstudio.com/}. I am no bash expert and really
only rely on it when I have no other option and/or am dealing with a
large number of files and don't want to open each of them. As a result,
this is just a quick overview to help you quickly understand how to set
up and run some simple scripts in bash. For background on bash/more
information I'd suggest our good friend Google or here:
\url{https://www.atlassian.com/git/tutorials/git-bash\#:~:text=Git\%20Bash\%20is\%20an\%20application,operating\%20system\%20through\%20written\%20commands.};
\url{https://en.wikipedia.org/wiki/Bash_(Unix_shell)}

\hypertarget{set-working-directory}{%
\section{Set Working Directory}\label{set-working-directory}}

To set your working directory, use the function ``cd'' and the location.
For example, to point


\end{document}
